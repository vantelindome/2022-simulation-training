\documentclass[a4paper]{jarticle}
\usepackage{amsmath}
\usepackage{titlesec}
\titleformat*{\section}{\large\bfseries}
\titleformat*{\subsection}{\normalsize\bfseries}

\renewcommand{\thesection}{問題\arabic{section}}
\renewcommand{\thesubsection}{(\arabic{subsection})}


\begin{document}

\title{シミュレーション実習 中間レポート}
\author{262201018 在田 陽一}
\date{2022/05/11}
\maketitle


% 問題1
\section{}
% (1)
\subsection{}
\noindent
条件より,棒の下半分が$y$軸と交わるためには,
\begin{equation}
     y_s \leq \frac{l}{2}\sin\theta \tag{1.1}
\end{equation}
であればよい。

\noindent
$0 \leq \theta \leq \frac{\pi}{2}$, $0 \leq y_s \leq \frac{d}{2}$ より,
この範囲において図(1.1)おいて$\theta$と$y_s$がなす点が境界を含む斜線部の範囲内に含まれる確率を求めればよい.

\noindent
よって確率$p$は,
\begin{equation}
    p \sim \frac{\int^\frac{\pi}{2}_0 \frac{l}{2}\sin \theta}{\frac{\pi}{2} \cdot \frac{d}{2}} \\
     =\frac{2l}{\pi d} \tag{1.2}
\end{equation}
となる.

\subsection{}
\noindent
式(1.2)より,
\begin{equation}
   \pi \sim  \frac{2l}{pd} \tag{1.3}
\end{equation}
と表せるので, $y_s$と$\theta$を乱数生成させることで1.1式と比較し,
それをもとに(1.3)式によって$\pi$を実測する.

\section{}
\subsection{}
\noindent
式(1)は定数係数の二階微分方程式より,$x=e^{\lambda x}$とおいて整理すると
\begin{equation}
    m\lambda^2 + \zeta\lambda + k = 0 \tag{2.1}
\end{equation}
という特性方程式(2.1)を得る.この解は
\begin{equation}
    \lambda = \frac{-\zeta + \sqrt{\zeta^2 - 4mk}}{2} \tag{2.2}
\end{equation}
であり,ルートの中身について場合分けすると,減衰振動,過減衰,臨界減衰を示す条件はそれぞれ
\begin{align*}
    \zeta & \geq \sqrt{4mk}   (減衰振動) \tag{2.3}\\
    \zeta & \leq \sqrt{4mk}   (過減衰) \tag{2.4}\\
    \zeta & = \sqrt{4mk}   (臨界減衰) \tag{2.5}\\
\end{align*}
となる.

\subsection{}
\noindent
式(1)について$x = \tilde{x}$, $t = t_0 \tilde{t}$, $\dot{x} = v = \frac{a}{t_0} \tilde{v}$
を代入して
\begin{equation}
    m \frac{a}{t_0} \dot{\tilde{v}}(\tilde{t}) = -\zeta \frac{a}{t_0} \tilde{v}(\tilde{t})
    -k a \tilde{x}(\tilde{t}) \tag{2.6}
\end{equation}
ここで
\begin{align*}
    \frac{d \tilde{v}(\tilde{t})}{dt} &= \frac{d \tilde{v}(\tilde{t})}{d \tilde{t}} \cdot \frac{d \tilde{t}}{dt} \\
    &= \frac{1}{t_0} \frac{d \tilde{v}(\tilde{t})}{d \tilde{t}} \tag{2.7}
\end{align*}
より,式(2.6)は
\begin{equation}
    \frac{m}{t_0^2} \frac{d \tilde{v}(\tilde{t})}{d \tilde{t}} = - \frac{\zeta}{t_0} \tilde{v}(\tilde{t})
    -k \tilde{x}(\tilde{t}) \tag{2.8}
\end{equation}
とあらわされる. 式(2.8)左辺について, 刻み幅$\Delta \tilde{t}$に関するオイラー法による離散化を実行すると
\begin{equation}
    \frac{d \tilde{v}(\tilde{t})}{d \tilde{t}} = \frac{\tilde{v}(\tilde{t} + \Delta \tilde{t}) - \tilde{v}(\tilde{t})}{\Delta \tilde{t}} 
    \tag {2.9}
\end{equation}
より, これを式(2.8)に代入して整理すると
\begin{align}
    \tilde{v}(\tilde{t} + \Delta \tilde{t}) = \left(1 - \frac{\zeta t_0 \Delta \tilde{t}}{m}\right) \tilde{v}(\tilde{t})
    - \frac{k t_0^2 \Delta \tilde{t}}{m} \tilde{x}(\tilde{t}) \tag{2.10}
\end{align}
また$\tilde{x}(\tilde{t} + \Delta \tilde{t})$について, 半陰的オイラー法に基づき
\begin{align*}
    \tilde{x}(\tilde{t} + \Delta \tilde{t}) &= \tilde{x}(\tilde{t}) + \tilde{v}(\tilde{t} + \Delta \tilde{t}) \Delta \tilde{t} \\
    &= \left(1 - \frac{k t_0^2 \Delta \tilde{t}}{m}\right) \tilde{x}(\tilde{t}) 
    + \left(1 - \frac{\zeta t_0 \Delta \tilde{t}}{m}\right) \tilde{v}(\tilde{t}) \Delta \tilde{t} \tag{2.11}
\end{align*}
以上より, 求める2項間漸化式は
\begin{subequations}
    \begin{align}    % equation 環境にしても同じ結果となる.
    \left\{
        \begin{aligned}
        & \tilde{v}(\tilde{t} + \Delta \tilde{t}) = \left(1 - \frac{\zeta t_0 \Delta \tilde{t}}{m}\right) \tilde{v}(\tilde{t})
        - \frac{k t_0^2 \Delta \tilde{t}}{m} \tilde{x}(\tilde{t})\\
        & \tilde{x}(\tilde{t} + \Delta \tilde{t} = \left(1 - \frac{k t_0^2 \Delta \tilde{t}}{m}\right) \tilde{x}(\tilde{t}) 
        + \left(1 - \frac{\zeta t_0 \Delta \tilde{t}}{m}\right) \tilde{v}(\tilde{t}) \Delta \tilde{t}
        \end{aligned}
    \right. \tag{2.12}
    \end{align}
\end{subequations}
となる.

\subsection{}
\noindent


\subsection{}
\noindent
\end{document}